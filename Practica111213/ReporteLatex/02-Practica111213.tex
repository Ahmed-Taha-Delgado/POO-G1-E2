\documentclass[letterpaper,12pt]{article}
\usepackage[top=1in, left=1.25in, right=1.25in, bottom=1in]{geometry}
\usepackage[utf8]{inputenc}
\usepackage[T1]{fontenc}
\usepackage[spanish]{babel}
\usepackage{graphicx}
\usepackage{caption}
\usepackage{float}
\usepackage[backend=biber,style=numeric,sorting=none]{biblatex}
\usepackage{amsmath}
\addbibresource{bib/referencias.bib}
\usepackage{subcaption}
\usepackage{fancyhdr}
\pagestyle{fancy}
\lhead{Reporte de la Práctica 11, 12 y 13}
\rhead{Programación Orientada a Objetos}

\begin{document}

\tableofcontents
\clearpage

\section{Introducción}

\begin{itemize}
\item \textbf{Planteamiento del Problema:} .

\item \textbf{Motivación:} .

\item \textbf{Objetivos:} .

\end{itemize}

\section{Marco Teórico}
\begin{itemize}
    \item \textbf{Archivos: } Los archivos, en términos computacionales, son conjuntos de datos con un nombre y extensión que permite almacenar información de manera permanente en memoria secundaria, para que pueda ser consultada, utilizada y modificada posteriormente tanto por el usuario como diferentes programas.

    Los programas se comunican con los archivos mediante \textit{flujos de datos}. Estos flujos se clasifican en \textbf{flujos de entrada} y \textbf{flujos de salida}. 
    \begin{itemize}
        \item \textbf{Flujos de entrada:} cuando los programas reciben información desde archivos o una fuente externa.
        \item \textbf{Flujos de salida:} cuando los programas envían esa información a un destino externo, como lo puede ser el usuario.
    \end{itemize} ~\cite{Archivos} ~\cite{Flujos}
    
    
\end{itemize}

\clearpage

\section{Desarrollo}

\subsection{Explicación del código de archivos:}

El código implementado en dart \textbf{main.dart} consta de un menú de opciones que nos permite entender y trabajar con las operaciones básicas de los archivos. El menú cuenta con 3 opciones que explicaremos a continuación y la opción salir. Cada opción manda a llamar a una función que permite realizar la operación ingresada. \\

\textbf{Opción 1. Crear archivo .txt y escribir texto:} Esta función le pide al usuario que ingrese un nombre para crear un archivo, si la cadena ingresada no esta vacía, crea ese nuevo archivo. Al crear el archivo, se le pide al usuario que ingrese el texto que contendrá el archivo, y para indicar que ya termino de escribir lo que se quiere guardar en el archivo, que escriba 'FIN'. El programa, mediante un ciclo \textit{while}, va almacenando las cadenas ingresadas por el usuario en una lista, y si se detecta una cadena 'FIN', se termina el ciclo. Finalmente con un \textit{try-catch}, se intentará crear un archivo con el nombre ingresado inicialmente por el usuario (creando un objeto de tipo \textbf{FILE} e indicando su nombre en el constructor), y con el método \textbf{writeAsStringSync} se escriben las cadenas guardadas en la lista de cadenas, dentro del archivo creado. En caso de no poder crear o escribir en el archivo, el bloque \textit{catch} muestra que ocurrio un error al guardar el archivo. \\

\textbf{Opción 2. Leer archivo existente:} Esta función le pide al usuario que ingrese el nombre o ruta del archivo que desea leer, si el nombre o ruta ingresado no se encuentra se termina esta funcion. Si la ejecución continua, con un \textit{try-catch} se intentará abrir un archivo con el nombre o ruta (indicándolo en el constructor), y leerlo con el método \textbf{readAsStringSync}, para posteriormente imprimirlo. \\

\textbf{Opción 3. Sobrescribir archivo existente:} Esta función le pide al usuario que ingrese el nombre o ruta del archivo a sobrescribir, si el nombre o ruta no se encuentra, se termina la funcion. Si el archivo se encuentra, se abre el archivo y se le pide confirmacion al usuario para sobrescribir el archivo, indicandole que esta accion borrara todo su contenido. En caso de aceptar la sobrescritura, de la misma manera que la primera opcion, se le pide al usuario que ingrese el texto y para terminar escriba 'FIN', y con un ciclo while se almacenan las cadenas en una lista, para que finalmente con un \textit{try-catch}, se intente sobrescribir el archivo con el método \textbf{writeAsStringSync}, agregando las cadenas guardadas en la lista al archivo.


\clearpage

\section{Diagramas UML}

\subsection{Diagramas del código de archivos:}

\begin{figure}[H]
    \centering
    \includegraphics[width=0.8\linewidth]{Imagenes/Clase UML Archivos.png}
    \caption*{Diagrama UML de Clases}
\end{figure}

\begin{figure}[H]
    \centering
    \includegraphics[width=0.8\linewidth]{Imagenes/Diagrama de secuencia archivos.png}
    \caption*{Diagrama UML de Secuencia}
\end{figure}

\section{Resultados}

\begin{figure}[H]
    \centering
    \includegraphics[width=0.49\linewidth]{Imagenes/Crear archivo.png}
    \includegraphics[width=0.49\linewidth]{Imagenes/Sobrescribir archivo.png}
    \includegraphics[width=0.49\linewidth]{Imagenes/Leer archivo.png}
    \caption*{Ejecución del código de archivos}
\end{figure}

\section{Conclusiones}


\printbibliography

\end{document}